
\chapter{OpenMP vs. MPI vs. PiP}

\section{Hello PiP}

\lstinputlisting[style=program,caption={Hello OMP},label=prg:hello-omp]{../prgs/hello-omp.c}
\begin{lstlisting}[style=example,caption={Output of ``Hello OMP''},label=out:hello-omp]
$ OMP_NUM_THREADS=5 ./hello-omp
Hello from OMP thread 3
Hello from OMP thread 4
Hello from OMP thread 2
Hello from OMP thread 0
Hello from OMP thread 1
\end{lstlisting}

\lstinputlisting[style=program,caption={Hello MPI},label=hello-mpi]{../prgs/hello-mpi.c}
\begin{lstlisting}[style=example,caption={Output of ``Hello MPI''},label=out:hello-mpi]
$ mpiexec -np 5 ./hello-mpi
Hello from MPI process 2
Hello from MPI process 3
Hello from MPI process 4
Hello from MPI process 1
Hello from MPI process 0
\end{lstlisting}

\lstinputlisting[style=program,caption={Hello PiP},label=prg:hello-pip]{../prgs/hello-pip.c}
\begin{lstlisting}[style=example,caption={Output of ``Hello PiP''},label=out:hello-pip]
$ piprun -n 5 ./hello-pip
Hello from PiP task 0
Hello from PiP task 1
Hello from PiP task 2
Hello from PiP task 3
Hello from PiP task 4
\end{lstlisting}

\section{Variables}


\lstinputlisting[style=program,caption={Vars0 OMP},label=prg:vars0-omp]{../prgs/vars0-omp.c}
\begin{lstlisting}[style=example,caption={Output of ``Vars0 OMP''},label=out:vars0-omp]
$ OMP_NUM_THREADS=5 ./vars0-omp
[1] gvar:0x601050 lvar:0x7ffd47c2bacc tid:0x7f9542221e4c
[4] gvar:0x601050 lvar:0x7ffd47c2bacc tid:0x7f9540a1ee4c
[0] gvar:0x601050 lvar:0x7ffd47c2bacc tid:0x7ffd47c2baac
[3] gvar:0x601050 lvar:0x7ffd47c2bacc tid:0x7f954121fe4c
[2] gvar:0x601050 lvar:0x7ffd47c2bacc tid:0x7f9541a20e4c
\end{lstlisting}

\lstinputlisting[style=program,caption={Vars0 MPI},label=vars0-mpi]{../prgs/vars0-mpi.c}
\begin{lstlisting}[style=example,caption={Output of ``Vars0 MPI''},label=out:vars0-mpi]
[1] gvar:0x601050 lvar:0x7ffdeec8d088
[2] gvar:0x601050 lvar:0x7ffd2e275998
[3] gvar:0x601050 lvar:0x7ffc163f6c58
[4] gvar:0x601050 lvar:0x7ffef9d2db88
[0] gvar:0x601050 lvar:0x7ffe438034d8
\end{lstlisting}

\lstinputlisting[style=program,caption={Vars0 PiP},label=prg:vars0-pip]{../prgs/vars0-pip.c}
\begin{lstlisting}[style=example,caption={Output of ``Vars0 PiP''},label=out:vars0-pip]
$ ../../bin/piprun -n 5 ./vars0-pip
[0] gvar:0x7f6a22849050 lvar:0x7f6a21c6e4f8
[1] gvar:0x7f6a21477050 lvar:0x7f6a2089c4f8
[2] gvar:0x7f6a1bfff050 lvar:0x7f6a1b4244f8
[3] gvar:0x7f6a1ac2d050 lvar:0x7f6a1a0524f8
[4] gvar:0x7f6a1985b050 lvar:0x7f6a18c804f8
\end{lstlisting}



\lstinputlisting[style=program,caption={Vars1 OMP},label=prg:vars1-omp]{../prgs/vars1-omp.c}
\begin{lstlisting}[style=example,caption={Output of ``Vars1 OMP''},label=out:vars1-omp]
$ OMP_NUM_THREADS=5 ./vars1-omp
[0] gvar=3
[4] gvar=3
[2] gvar=3
[3] gvar=3
[1] gvar=3
$ OMP_NUM_THREADS=5 ./vars1-omp
[1] gvar=4
[2] gvar=4
[0] gvar=4
[3] gvar=4
[4] gvar=4
\end{lstlisting}

\lstinputlisting[style=program,caption={Vars1 MPI},label=vars1-mpi]{../prgs/vars1-mpi.c}
\begin{lstlisting}[style=example,caption={Output of ``Vars1 MPI''},label=out:vars1-mpi]
$ mpirun -np 5 ./vars1-mpi
[0] gvar=0
[1] gvar=1
[2] gvar=2
[3] gvar=3
[4] gvar=4
$ mpirun -np 5 ./vars1-mpi
[0] gvar=0
[1] gvar=1
[2] gvar=2
[3] gvar=3
[4] gvar=4
$ mpirun -np 5 ./vars1-mpi
[0] gvar=0
[1] gvar=1
[2] gvar=2
[3] gvar=3
[4] gvar=4
\end{lstlisting}

\lstinputlisting[style=program,caption={Vars1 PiP},label=prg:vars1-pip]{../prgs/vars1-pip.c}
\begin{lstlisting}[style=example,caption={Output of ``Vars1 PiP''},label=out:vars1-pip]
$ piprun -n 5 ./vars1-pip
[4] gvar=4
[2] gvar=2
[3] gvar=3
[0] gvar=0
[1] gvar=1
$ piprun -n 5 ./vars1-pip
[3] gvar=3
[2] gvar=2
[4] gvar=4
[0] gvar=0
[1] gvar=1
$ piprun -n 5 ./vars1-pip
[4] gvar=4
[3] gvar=3
[1] gvar=1
[2] gvar=2
[0] gvar=0
\end{lstlisting}


\lstinputlisting[style=program,caption={Vars2 PiP},label=prg:vars2-pip]{../prgs/vars2-pip.c}
\begin{lstlisting}[style=example,caption={Output of ``Vars2 PiP''},label=out:vars2-pip]
$ piprun -n 5 ./vars2-pip
[4] 1st: gvar=4
[2] 1st: gvar=2
[1] 1st: gvar=1
[0] 1st: gvar=0
[3] 1st: gvar=3
[0] 2nd: gvar=100
$ piprun -n 5 ./vars2-pip
[0] 1st: gvar=0
[4] 1st: gvar=4
[3] 1st: gvar=3
[1] 1st: gvar=1
[2] 1st: gvar=2
[0] 2nd: gvar=104
\end{lstlisting}

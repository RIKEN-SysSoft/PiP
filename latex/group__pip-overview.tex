\hypertarget{group__pip-overview}{\section{pip-\/overview}
\label{group__pip-overview}\index{pip-\/overview@{pip-\/overview}}
}


the Pi\-P library  


the Pi\-P library \hypertarget{group__pip-overview_overview}{}\subsection{Overview}\label{group__pip-overview_overview}
Pi\-P is a user-\/level library which allows a process to create sub-\/processes into the same virtual address space where the parent process runs. The parent process and sub-\/processes share the same address space, however, each process has its own variables. So, each process runs independently from the other process. If some or all processes agreed, then data own by a process can be accessed by the other processes.

Those processes share the same address space, just like pthreads, and each process has its own variables like a process. The parent process is called {\itshape Pi\-P} {\itshape process} and sub-\/processes are called {\itshape Pi\-P} {\itshape task} since it has the best of the both worlds of processes and pthreads.

Pi\-P root can spawn one or more number of Pi\-P tasks. The Pi\-P root and Pi\-P tasks shared the same address space. The executable of the Pi\-P task must be compiled (with the \char`\"{}-\/fpie\char`\"{} compile option) and linked (with the \char`\"{}-\/pie\char`\"{} linker option) as P\-I\-E (Position Independent Executable).

When a Pi\-P root or Pi\-P task wants to be accessed the its own data by the other(s), firstly a memory region where the data to be accessed are located must be {\itshape exported}. Then the exported memory region is {\itshape imported} so that the exported and imported data can be accessed. The Pi\-P library supports the functions to export and import the memory region to be accessible.\hypertarget{group__pip-overview_PiP}{}\subsection{User-\/\-Level Process (\-U\-L\-P)}\label{group__pip-overview_PiP}
Pi\-P also supports U\-L\-Ps which are explicitly scheduled by Pi\-P tasks. Unlike the other user-\/level thread libraries, Pi\-P U\-L\-Ps can run with different programs. Due to the G\-L\-I\-B\-C constraints, U\-L\-Ps can be created by the Pi\-P root process. The created U\-L\-Ps are associated with a Pi\-P task by specifying U\-L\-Ps when the task is created by calling {\bfseries \hyperlink{group__libpip_ga8055012fb65183ce17ad8cd8da292d54}{pip\-\_\-task\-\_\-spawn()}}. A U\-L\-P can yield, suspend, and resume its execution by calling Pi\-P U\-L\-P functions. A Pi\-P task will be terminated only when all U\-L\-Ps scheduled by the task terminate.\hypertarget{group__pip-overview_execution-mode}{}\subsection{Execution mode}\label{group__pip-overview_execution-mode}
There are several Pi\-P implementation modes which can be selected at the runtime. These implementations can be categorized into two according to the behavior of Pi\-P tasks,


\begin{DoxyItemize}
\item {\ttfamily Pthread}, and
\item {\ttfamily Process}.
\end{DoxyItemize}

In the pthread mode, although each Pi\-P task has its own variables unlike thread, Pi\-P task behaves more like Pthread, having a T\-I\-D, having the same file descriptor space, having the same signal delivery semantics as Pthread does, and so on. In the process mode, Pi\-P task behaves more like a process, having a P\-I\-D, having an independent file descriptor space, having the same signal delivery semantics as Linux process does, and so on.

When the {\ttfamily P\-I\-P\-\_\-\-M\-O\-D\-E} environment variable set to "thread" then the Pi\-P library runs based on the pthread mode, and it is set to "process" then it runs with the process mode. There are also two implementations in the {\bfseries process} mode; "process\-:preload" and "process\-:pipclone" The former one must be with the {\bfseries L\-D\-\_\-\-P\-R\-E\-L\-O\-A\-D} environment variable setting so that the {\bfseries clone()} system call wrapper can work with. The latter one can only be specified with the P\-I\-P-\/patched glibc library (see below\-: {\bfseries G\-L\-I\-B\-C} issues).

There several function provided by the Pi\-P library to absorb the difference due to the execution mode\hypertarget{group__pip-overview_limitation}{}\subsection{Limitation}\label{group__pip-overview_limitation}
Pi\-P allows Pi\-P root and Pi\-P tasks to share the data, so the function pointer can be passed to the others. However, jumping into the code owned by the other may not work properly for some reasons.\hypertarget{group__pip-overview_compile-and-link}{}\subsection{Compile and Link User programs}\label{group__pip-overview_compile-and-link}
The Pi\-P root ust be linked with the Pi\-P library and libpthread. The programs able to run as a Pi\-P task must be compiled with the "-\/fpie" compile option and the "-\/pie -\/rdynamic" linker options.\hypertarget{group__pip-overview_glibc-issues}{}\subsection{G\-L\-I\-B\-C issues}\label{group__pip-overview_glibc-issues}
The Pi\-P library is implemented at the user-\/level, i.\-e. no need of kernel patches nor kernel modules. Due to the novel usage of combining {\ttfamily dlmopn()} G\-L\-I\-B\-C function and {\ttfamily clone()} systemcall, there are some issues found in the G\-L\-I\-B\-C. To avoid this issues, Pi\-P users may have the patched G\-L\-I\-B\-C provided by the Pi\-P development team.\hypertarget{group__pip-overview_gdb-issue}{}\subsection{G\-D\-B issue}\label{group__pip-overview_gdb-issue}
The normal gdb debugger only works with the Pi\-P root. Pi\-P-\/aware G\-D\-B is also provided and must be used for debugging Pi\-P tasks.\hypertarget{group__pip-overview_author}{}\subsection{Author}\label{group__pip-overview_author}
Atsushi Hori (R\-I\-K\-E\-N, Japan) \href{mailto:ahori@riken.jp}{\tt ahori@riken.\-jp} 
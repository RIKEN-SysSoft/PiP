Spawning a Pi\-P task

\begin{DoxyParagraph}{Synopsis\-:}
\#include $<$\hyperlink{pip_8h_source}{pip.\-h}$>$ \par
int pip\-\_\-task\-\_\-spawn( pip\-\_\-spawn\-\_\-program\-\_\-t $\ast$progp, uint32\-\_\-t coreno, uint32\-\_\-t opts, int $\ast$pipidp, pip\-\_\-spawn\-\_\-hook\-\_\-t $\ast$hookp );
\end{DoxyParagraph}
\begin{DoxyParagraph}{Description\-:}
This function spawns a Pi\-P task specified by {\ttfamily progp}. 
\end{DoxyParagraph}
\begin{DoxyParagraph}{}
In the process execution mode, the file descriptors having the {\ttfamily F\-D\-\_\-\-C\-L\-O\-E\-X\-E\-C} flag is closed and will not be passed to the spawned Pi\-P task. This simulated close-\/on-\/exec will not take place in the pthread execution mode.
\end{DoxyParagraph}

\begin{DoxyParams}[1]{Parameters}
\mbox{\tt in}  & {\em progp} & Program information to spawn as a Pi\-P task \\
\hline
\mbox{\tt in}  & {\em coreno} & Core number for the Pi\-P task to be bound to. If {\ttfamily P\-I\-P\-\_\-\-C\-P\-U\-C\-O\-R\-E\-\_\-\-A\-S\-I\-S} is specified, then the core binding will not take place. \\
\hline
\mbox{\tt in}  & {\em opts} & option flags \\
\hline
\mbox{\tt in,out}  & {\em pipidp} & Specify Pi\-P I\-D of the spawned Pi\-P task. If {\ttfamily P\-I\-P\-\_\-\-P\-I\-P\-I\-D\-\_\-\-A\-N\-Y} is specified, then the Pi\-P I\-D of the spawned Pi\-P task is up to the Pi\-P library and the assigned Pi\-P I\-D will be returned. \\
\hline
\mbox{\tt in}  & {\em hookp} & Hook information to be invoked before and after the program invokation.\\
\hline
\end{DoxyParams}
\begin{DoxyReturn}{Returns}
Zero is returned if this function succeeds. On error, an error number is returned. 
\end{DoxyReturn}

\begin{DoxyRetVals}{Return values}
{\em E\-P\-E\-R\-M} & Pi\-P library is not yet initialized \\
\hline
{\em E\-P\-E\-R\-M} & Pi\-P task tries to spawn child task \\
\hline
{\em E\-I\-N\-V\-A\-L} & {\ttfamily progp} is {\ttfamily N\-U\-L\-L} \\
\hline
{\em E\-I\-N\-V\-A\-L} & {\ttfamily opts} is invalid and/or unacceptable \\
\hline
{\em E\-I\-N\-V\-A\-L} & the value off {\ttfamily pipidp} is invalid \\
\hline
{\em E\-B\-U\-S\-Y} & specified Pi\-P I\-D is alredy occupied \\
\hline
{\em E\-N\-O\-M\-E\-M} & not enough memory \\
\hline
{\em E\-N\-X\-I\-O} & {\ttfamily dlmopen} failss\\
\hline
\end{DoxyRetVals}
\begin{DoxyNote}{Note}
In the process execution mode, each Pi\-P task may have its own file descriptors, signal handlers, and so on, just like a process. Contrastingly, in the pthread executionn mode, file descriptors and signal handlers are shared among Pi\-P root and Pi\-P tasks while maintaining the privatized variables.
\end{DoxyNote}
\begin{DoxyParagraph}{Bugs\-:}
In theory, there is no reason to restrict for a Pi\-P task to spawn another Pi\-P task. However, the current glibc implementation does not allow to do so. 
\end{DoxyParagraph}
\begin{DoxyParagraph}{}
If the root process is multithreaded, only the main thread can call this function.
\end{DoxyParagraph}
\begin{DoxySeeAlso}{See Also}
pip\-\_\-task\-\_\-spawn(3), pip\-\_\-spawn\-\_\-from\-\_\-main(3), pip\-\_\-spawn\-\_\-from\-\_\-func(3), pip\-\_\-spawn\-\_\-hook(3) 
\end{DoxySeeAlso}

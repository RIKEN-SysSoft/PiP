Setting information to invoke a Pi\-P task starting from a function defined in a program

\begin{DoxyParagraph}{Synopsis\-:}
\#include $<$\hyperlink{pip_8h_source}{pip.\-h}$>$ \par
pip\-\_\-spawn\-\_\-from\-\_\-func( pip\-\_\-spawn\-\_\-program\-\_\-t $\ast$progp, char $\ast$prog, char $\ast$funcname, void $\ast$arg, char $\ast$$\ast$envv, void $\ast$exp );
\end{DoxyParagraph}
\begin{DoxyParagraph}{Description\-:}
This function sets the required information to invoke a program, starting from the {\ttfamily main()} function. The function should have the function prototype as shown below; 
\begin{DoxyCode}
\textcolor{keywordtype}{int} start\_func( \textcolor{keywordtype}{void} *arg )
\end{DoxyCode}
 This start function must be globally defined in the program.. The returned integer of the start function will be treated in the same way as the {\ttfamily main} function. This implies that the {\ttfamily pip\-\_\-wait} function family called from the Pi\-P root can retrieve the return code.
\end{DoxyParagraph}

\begin{DoxyParams}[1]{Parameters}
\mbox{\tt out}  & {\em progp} & Pointer to the {\ttfamily pip\-\_\-spawn\-\_\-program\-\_\-t} structure in which the program invokation information will be set \\
\hline
\mbox{\tt in}  & {\em prog} & Path to the executiable file. \\
\hline
\mbox{\tt in}  & {\em funcname} & Function name to be started \\
\hline
\mbox{\tt in}  & {\em arg} & Argument which will be passed to the start function \\
\hline
\mbox{\tt in}  & {\em envv} & Environment variables. If this is {\ttfamily N\-U\-L\-L}, then the {\ttfamily environ} variable is used for the spawning Pi\-P task. \\
\hline
\mbox{\tt in}  & {\em exp} & Export value to the spawning Pi\-P task\\
\hline
\end{DoxyParams}
\begin{DoxySeeAlso}{See Also}
pip\-\_\-task\-\_\-spawn(3), pip\-\_\-spawn\-\_\-from\-\_\-main(3) 
\end{DoxySeeAlso}

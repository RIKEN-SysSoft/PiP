\section*{Process-\/in-\/\-Process (Pi\-P)}

Pi\-P is a user-\/level library to have the best of the both worlds of multi-\/process and multi-\/thread parallel execution models. Pi\-P allows a process to create sub-\/processes into the same virtual address space where the parent process runs. The parent process and sub-\/processes share the same address space, however, each process has its own variable set. So, each process runs independently from the other process. If some or all processes agree, then data owned by a process can be accessed by the other processes. Those processes share the same address space, just like pthreads, but each process has its own variables like the process execution model. Hereinafter, the parent process is called Pi\-P process and a sub-\/process are called a Pi\-P task.

\subsection*{Pi\-P Versions}

Currently there are three Pi\-P library versions\-:


\begin{DoxyItemize}
\item Version 1 -\/ Deprecated
\item Version 2 -\/ Stable version
\item Version 3 -\/ Stable version supporting B\-L\-T and U\-L\-P (experimental)
\end{DoxyItemize}

Unfortunately each version has unique A\-B\-I and there is no A\-B\-I compatibility among them. The functionality of Pi\-P-\/v1 is almost the same with Pi\-P-\/v2, however, Pi\-P-\/v2's A\-P\-I is a subset of the Pi\-P-\/v3's A\-P\-I. Hereafter {\bfseries N} denotes the Pi\-P version number.

\subsection*{Bi-\/\-Level Thread (B\-L\-T, from v3)}

Pi\-P also provides new thread implementation named \char`\"{}\-Bi-\/\-Level Thread
(\-B\-L\-T)\char`\"{}, again, to take the best of two worlds, Kernel-\/\-Level Thread (K\-L\-T) and User-\/\-Level Thread (U\-L\-T) here. A B\-L\-T is a Pi\-P task. When a Pi\-P task is created it runs as a K\-L\-T. At any point the K\-L\-T can becomme a U\-L\-T by decoupling the associated kernel thread from the K\-L\-T. The decoupled kernel thread becommes idle. Later, the U\-L\-T can become K\-L\-T again by coupling with the kernel thread.

\subsection*{User-\/\-Level Process (U\-L\-P, from v3)}

As described, Pi\-P allows Pi\-P tasks to share the same virtual address space. This mans that a Pi\-P task can context-\/switch to the other Pi\-P task at user-\/level. This is called User-\/\-Level Process where processes may be derived from the same program or different programs. Threads basically share most of the kernel resources, such as address space, file descriptors, a process id, and so on whilst processes do not. Every process has its ows file descriptor space, for example. When a U\-L\-P is scheduled by a K\-L\-T having P\-I\-D 1000, then the getpid() is called by the U\-L\-P returns 1000. Further, when the U\-L\-T is migrated to be scheduled by the other K\-L\-T, then the returned P\-I\-D is different. So, when implemnting a U\-L\-P system, this systemcall consistency must be preserved. In U\-L\-P on Pi\-P, the consistency can be maintained by utilizing the above B\-L\-T mechanism. When a U\-L\-T tries to call a system call, it is coupled with its kernel thread which was created at the beginning as a K\-L\-T. It should be note that Thread Local Storage (T\-L\-S) regions are also switched when switching U\-L\-P (and B\-L\-T) contexts.

\subsection*{Execution Mode}

There are several Pi\-P implementation modes which can be selected at the runtime. These implementations can be categorized into two;


\begin{DoxyItemize}
\item Process and
\item (P)Thread.
\end{DoxyItemize}

In the pthread mode, although each Pi\-P task has its own static variables unlike thread, Pi\-P task behaves more like P\-Thread, having a T\-I\-D, having the same file descriptor space, having the same signal delivery semantics as Pthread does, and so on. In the process mode, a Pi\-P task behaves more like a process, having a P\-I\-D, having an independent file descriptor space, having the same signal delivery semantics as Linux process does, and so on. The above mentioned U\-L\-P can only work with the process mode.

When the {\ttfamily P\-I\-P\-\_\-\-M\-O\-D\-E} environment variable set to \char`\"{}()thread\char`\"{} then the Pi\-P library runs in the pthread mode, and if it is set to \char`\"{}process\char`\"{} then it runs in the process mode. There are also three implementations in the process mode; \char`\"{}process\-:preload,\char`\"{} \char`\"{}process\-:pipclone\char`\"{} and \char`\"{}process\-:got.\char`\"{} The \char`\"{}process\-:preload\char`\"{} mode must be with the L\-D\-\_\-\-P\-R\-E\-L\-O\-A\-D environment variable setting so that the clone() system call wrapper can work with. The \char`\"{}process\-:pipclone\char`\"{} mode is only effective with the P\-I\-P-\/patched glibc library (see below).

Several function are made available by the Pi\-P library to absorb the functional differences due to the execution modes.

\section*{License}

This package is licensed under the 2-\/clause simplified B\-S\-D License -\/ see the \mbox{[}L\-I\-C\-E\-N\-S\-E\mbox{]}(L\-I\-C\-E\-N\-S\-E) file for details.

\section*{Installation}

There are several ways to install Pi\-P; Docker, Spack, R\-P\-M, and building fromm the source code.

\subsection*{Docker image}

Download and run the Pi\-P Docker image. \begin{DoxyVerb}$ docker pull rikenpip/pip-vN
$ sudo docker run -it rikenpip/pip-vN /bin/bash
\end{DoxyVerb}


\subsection*{\href{https://spack.readthedocs.io/}{\tt Spack}}

Download spack and do the follwoing; \begin{DoxyVerb}$ git clone https://github.com/spack/spack.git
$ cd spack/bin
$ spack install process-in-process
\end{DoxyVerb}


\subsection*{Installation from R\-P\-Ms}

R\-P\-M packages and their yum repository are also available for Cent\-O\-S 7 / R\-H\-E\-L7. \begin{DoxyVerb}$ sudo rpm -Uvh https://git.sys.r-ccs.riken.jp/PiP/package/el/7/noarch/pip-1/pip-release-N-0.noarch.rpm
$ sudo yum install pip-glibc
$ sudo yum install pip pip-debuginfo
$ sudo yum install pip-gdb
\end{DoxyVerb}


If Pi\-P packages are installed by the above R\-P\-Ms, {\bfseries P\-I\-P\-\_\-\-I\-N\-S\-T\-A\-L\-L\-\_\-\-D\-I\-R} will be \char`\"{}/usr.\char`\"{}

\subsection*{Source Code}

The installation of Pi\-P related packages must follow the order below;


\begin{DoxyEnumerate}
\item Build Pi\-P-\/glibc (optional)
\item Build Pi\-P
\item Build Pi\-P-\/gdb (optional)
\end{DoxyEnumerate}

By using Pi\-P-\/glibc, users can create up to 300 Pi\-P tasks which can be dbugged by using Pi\-P-\/gdb. In other words, without installing Pi\-P-\/glibc, users can create up to around 10 Pi\-P tasks (the number depends on the program) and cannot debug by using Pi\-P-\/gdb. Above Docker image contains Pi\-P-\/glibc and Pi\-P-\/gdb, and the S\-Pack recipe installs Pi\-P-\/glibc and Pi\-P-\/gdb additionally.


\begin{DoxyItemize}
\item \href{https://github.com/RIKEN-SysSoft/PiP-glibc}{\tt Pi\-P-\/glibc} -\/ patched G\-N\-U libc for Pi\-P
\item \href{https://github.com/RIKEN-SysSoft/PiP}{\tt Pi\-P} -\/ Process in Process (this package)
\item \href{https://github.com/RIKEN-SysSoft/PiP-gdb}{\tt Pi\-P-\/gdb} -\/ patched gdb to debug Pi\-P root and Pi\-P tasks.
\end{DoxyItemize}

Before installing Pi\-P, we strongly recommend you to install Pi\-P-\/glibc. After installing Pi\-P, Pi\-P-\/gdb can be installed.

\subsection*{Installation from the source code.}

In addition to the above three Pi\-P related packages, there is Pi\-P installing program.


\begin{DoxyItemize}
\item \href{https://github.com/RIKEN-SysSoft/PiP-pip}{\tt Pi\-P-\/pip} -\/ Pi\-P package installing program
\end{DoxyItemize}

This is the easiest way to install Pi\-P packages from the source code is using the {\ttfamily pip-\/pip} command. This command clones source codes from the G\-I\-T\-H\-U\-B repos, build and install them including Pi\-P documents. Here is the usage of Pi\-P-\/pip command; \begin{DoxyVerb}$ git clone https://github.com/RIKEN-SysSoft/PiP-pip.git
$ cd PiP-pip
$ ./pip-pip --pip=PIP_VERSION --build=BUILD_DIR --prefix=INSTALL_DIR
\end{DoxyVerb}


\section*{Pi\-P Documents}

The following Pi\-P documents are created by using \href{https://www.doxygen.nl/}{\tt Doxygen}.

\subsection*{Man pages}

Man pages will be installed at {\bfseries P\-I\-P\-\_\-\-I\-N\-S\-T\-A\-L\-L\-\_\-\-D\-I\-R}/share/man. \begin{DoxyVerb}$ man -M PIP_INSTALL_DIR/share/man 7 libpip
\end{DoxyVerb}


Or, use the pip-\/man command (from v2). \begin{DoxyVerb}$ PIP_INSTALL_DIR/bin/pip-man 7 libpip
\end{DoxyVerb}


The above two exammples will show you the same document you are reading.

\subsection*{P\-D\-F}

P\-D\-F documents will be installed at {\bfseries P\-I\-P\-\_\-\-I\-N\-S\-T\-A\-L\-L\-\_\-\-D\-I\-R}/share/pdf.

\subsection*{H\-T\-M\-L}

H\-T\-M\-L documents will be installed at {\bfseries P\-I\-P\-\_\-\-I\-N\-S\-T\-A\-L\-L\-\_\-\-D\-I\-R}/share/html.

\section*{Getting Started}

\subsection*{Compile and link your Pi\-P programs}


\begin{DoxyItemize}
\item pipcc(1) command (since v2)
\end{DoxyItemize}

You can use pipcc(1) command to compile and link your Pi\-P programs. \begin{DoxyVerb}$ pipcc -Wall -O2 -g -c pip-prog.c
$ pipcc -Wall -O2 -g -o pip-prog pip-prog.c
\end{DoxyVerb}


\subsection*{Run your Pi\-P programs}


\begin{DoxyItemize}
\item pip-\/exec(1) command (piprun(1) in Pi\-P v1)
\end{DoxyItemize}

Let's assume that you have a non-\/\-Pi\-P program(s) and wnat to run as Pi\-P tasks. All you have to do is to compile your program by using the above pipcc(1) command and to use the pip-\/exec(1) command to run your program as Pi\-P tasks. \begin{DoxyVerb}$ pipcc myprog.c -o myprog
$ pip-exec -n 8 ./myprog
$ ./myprog
\end{DoxyVerb}


In this case, the pip-\/exec(1) command becomes the Pi\-P root and your program runs as 8 Pi\-P tasks. Note that the 'myprog.\-c' may or may not call any Pi\-P functions. Your program can also run as a normal program (not as a Pi\-P task) without using the pip-\/exec(1) command.

You may write your own Pi\-P programs whcih includes the Pi\-P root programming. In this case, your program can run without using the pip-\/exec(1) command.

If you get the following message when you try to run your program; \begin{DoxyVerb}PiP-ERR(19673) './myprog' is not PIE
\end{DoxyVerb}


Then this means that the 'myprog' is not compiled by using the pipcc(1) command properly. You may check if your program(s) can run as a Pi\-P root and/or Pi\-P task by using the pip-\/check(1) command (from v2); \begin{DoxyVerb}$ pip-check a.out
a.out : Root&Task
\end{DoxyVerb}


Above example shows that the 'a.\-out' program can run as a Pi\-P root and Pi\-P tasks.


\begin{DoxyItemize}
\item pips(1) command (from v2)

You can see how your Pi\-P program is running in realtimme by using the pips(1) command.
\end{DoxyItemize}

List the Pi\-P tasks via the 'ps' command; \begin{DoxyVerb}$ pips -l [ COMMAND ]
\end{DoxyVerb}


or, show the activities of Pi\-P tasks via the 'top' command; \begin{DoxyVerb}$ pips -t [ COMMAND ]
\end{DoxyVerb}


Here {\bfseries C\-O\-M\-M\-A\-N\-D} is the name (not a path) of Pi\-P program you are running.

Additionally you can kill all of your Pi\-P tasks by using the same pips(1) command; \begin{DoxyVerb}$ pips -s KILL [ COMMAND ]
\end{DoxyVerb}


\subsection*{Debugging your Pi\-P programs by the pip-\/gdb command}

The following procedure attaches all Pi\-P tasks and Pi\-P root which created those tasks. Each Pi\-P 'processes' is treated as a G\-D\-B inferior in Pi\-P-\/gdb. \begin{DoxyVerb}$ pip-gdb
(gdb) attach PID
\end{DoxyVerb}


The attached inferiors can be seen by the following G\-D\-B command\-: \begin{DoxyVerb}(gdb) info inferiors
  Num  Description              Executable
  4    process 6453 (pip 2)     /somewhere/pip-task-2
  3    process 6452 (pip 1)     /somewhere/pip-task-1
  2    process 6451 (pip 0)     /somewhere/pip-task-0
* 1    process 6450 (pip root)  /somewhere/pip-root
\end{DoxyVerb}


You can select and debug an inferior by the following G\-D\-B command\-: \begin{DoxyVerb}(gdb) inferior 2
[Switching to inferior 2 [process 6451 (pip 0)] (/somewhere/pip-task-0)]
\end{DoxyVerb}


When an already-\/attached program calls '\hyperlink{group__PiP-1-spawn_gae9187ea22ecf0623fa3ecfba5337f52d}{pip\-\_\-spawn()}' and becomes a Pi\-P root task, the newly created Pi\-P child tasks aren't attached automatically, but you can add empty inferiors and then attach the Pi\-P child tasks to the inferiors. e.\-g. \begin{DoxyVerb}.... type Control-Z to stop the root task.
^Z
Program received signal SIGTSTP, Stopped (user).

(gdb) add-inferior
Added inferior 2
(gdb) inferior 2
(gdb) attach 1902

(gdb) add-inferior
Added inferior 3
(gdb) inferior 3
(gdb) attach 1903

(gdb) add-inferior
Added inferior 4
(gdb) inferior 4
(gdb) attach 1904

(gdb) info inferiors
  Num  Description              Executable
* 4    process 1904 (pip 2)     /somewhere/pip-task-2
  3    process 1903 (pip 1)     /somewhere/pip-task-1
  2    process 1902 (pip 0)     /somewhere/pip-task-0
  1    process 1897 (pip root)  /somewhere/pip-root
\end{DoxyVerb}


You can attach all relevant Pi\-P tasks by\-: \begin{DoxyVerb}$ pip-gdb -p PID-of-your-PiP-program
\end{DoxyVerb}


(from v2)

If the P\-I\-P\-\_\-\-G\-D\-B\-\_\-\-P\-A\-T\-H environment is set to the path pointing to Pi\-P-\/gdb executable file, then Pi\-P-\/gdb is automatically attached when an excetion signal (S\-I\-G\-S\-E\-G\-V and S\-I\-G\-H\-U\-P by default) is delivered. The exception signals can also be defined by setting the P\-I\-P\-\_\-\-G\-D\-B\-\_\-\-S\-I\-G\-N\-A\-L\-S environment. Signal names (case insensitive) can be concatenated by the '+' or '-\/' symbol. 'all' is reserved to specify most of the signals. For example, 'A\-L\-L-\/\-T\-E\-R\-M' means all signals excepting S\-I\-G\-T\-E\-R\-M, another example, 'P\-I\-P\-E+\-I\-N\-T' means S\-I\-G\-P\-I\-P\-E and S\-I\-G\-I\-N\-T. If one of the specified or default signals is delivered, then Pi\-P-\/gdb will be attached automatically. The Pi\-P-\/gdb will show backtrace by default. If users specify P\-I\-P\-\_\-\-G\-D\-B\-\_\-\-C\-O\-M\-M\-A\-N\-D that a filename containing some G\-D\-B commands, then those G\-D\-B commands will be executed by Pi\-P-\/gdb, instead of backtrace, in batch mode. If the P\-I\-P\-\_\-\-S\-T\-O\-P\-\_\-\-O\-N\-\_\-\-S\-T\-A\-R\-T environment is set (to any value), then the Pi\-P library delivers S\-I\-G\-S\-T\-O\-P to a spawned Pi\-P task which is about to start user program.

\section*{Mailing List}

\href{mailto:pip@ml.riken.jp}{\tt pip@ml.\-riken.\-jp}

\section*{Publications}

\subsection*{Research papers}

Atsushi Hori, Min Si, Balazs Gerofi, Masamichi Takagi, Jay Dayal, Pavan Balaji, and Yutaka Ishikawa. \char`\"{}\-Process-\/in-\/process\-: techniques for
practical address-\/space sharing,\char`\"{} In Proceedings of the 27th International Symposium on High-\/\-Performance Parallel and Distributed Computing (H\-P\-D\-C '18). A\-C\-M, New York, N\-Y, U\-S\-A, 131-\/143. D\-O\-I\-: \href{https://doi.org/10.1145/3208040.3208045}{\tt https\-://doi.\-org/10.\-1145/3208040.\-3208045}

Atsushi Hori, Balazs Gerofi, and Yuataka Ishikawa. \char`\"{}\-An Implementation
of User-\/\-Level Processes using Address Space Sharing,\char`\"{} 2020 I\-E\-E\-E International Parallel and Distributed Processing Symposium Workshops (I\-P\-D\-P\-S\-W), New Orleans, L\-A, U\-S\-A, 2020, pp. 976-\/984, D\-O\-I\-: \href{https://doi.org/10.1109/IPDPSW50202.2020.00161}{\tt https\-://doi.\-org/10.\-1109/\-I\-P\-D\-P\-S\-W50202.\-2020.\-00161}.

Kaiming Ouyang, Min Si, Atsushi Hori, Zizhong Chen and Pavan Balaji. \char`\"{}\-C\-A\-B-\/\-M\-P\-I\-: Exploring Interprocess Work Stealing toward Balanced
\-M\-P\-I Communication,\char`\"{} in S\-C’20 (to appear)

\subsection*{Presentation Slides}


\begin{DoxyItemize}
\item \mbox{[}H\-P\-D\-C'18\mbox{]} \href{file:/home/ahori/PiP/PiP-pip/install/x86_64_redhat-7_pip-2/pip-install/share/slides/HPDC18.pdf}{\tt file\-:/home/ahori/\-Pi\-P/\-Pi\-P-\/pip/install/x86\-\_\-64\-\_\-redhat-\/7\-\_\-pip-\/2/pip-\/install/share/slides/\-H\-P\-D\-C18.\-pdf}
\item \mbox{[}R\-O\-S\-S'18\mbox{]} \href{file:/home/ahori/PiP/PiP-pip/install/x86_64_redhat-7_pip-2/pip-install/share/slides/HPDC18-ROSS.pdf}{\tt file\-:/home/ahori/\-Pi\-P/\-Pi\-P-\/pip/install/x86\-\_\-64\-\_\-redhat-\/7\-\_\-pip-\/2/pip-\/install/share/slides/\-H\-P\-D\-C18-\/\-R\-O\-S\-S.\-pdf}
\item \mbox{[}I\-P\-D\-P\-S/\-R\-A\-D\-R'20\mbox{]} \href{file:/home/ahori/PiP/PiP-pip/install/x86_64_redhat-7_pip-2/pip-install/share/slides/IPDPS-RSADR-2020.pdf}{\tt file\-:/home/ahori/\-Pi\-P/\-Pi\-P-\/pip/install/x86\-\_\-64\-\_\-redhat-\/7\-\_\-pip-\/2/pip-\/install/share/slides/\-I\-P\-D\-P\-S-\/\-R\-S\-A\-D\-R-\/2020.\-pdf} \section*{Author}
\end{DoxyItemize}

Atsushi Hori Riken Center for Commputational Science (R-\/\-C\-C\-S) Japan 
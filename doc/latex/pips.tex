List or kill running Pi\-P tasks

\begin{DoxyParagraph}{Synopsis}
pips \mbox{[}a\mbox{]}\mbox{[}u\mbox{]}\mbox{[}x\mbox{]} \mbox{[}P\-I\-P-\/\-O\-P\-T\-I\-O\-N\-S\mbox{]} \mbox{[}-\/\mbox{]} \mbox{[}P\-A\-T\-T\-E\-R\-N ..\mbox{]}
\end{DoxyParagraph}

\begin{DoxyParams}{Parameters}
{\em a$|$u$|$x} & similar to the aux options of the Linux ps command \\
\hline
{\em -\/-\/root} & List Pi\-P root(s) \\
\hline
{\em -\/-\/task} & List Pi\-P task(s) \\
\hline
{\em -\/-\/family} & List Pi\-P root(s) and Pi\-P task(s) in family order \\
\hline
{\em -\/-\/kill} & Send S\-I\-G\-T\-E\-R\-M to Pi\-P root(s) and task(s) \\
\hline
{\em -\/-\/signal} & Send a signal to Pi\-P root(s) and task(s). This option must be followed by a signal number of name. \\
\hline
{\em -\/-\/ps} & Run the ps Linux command. This option may have {\ttfamily ps} command option(s) separated by comma (,) \\
\hline
{\em -\/-\/top} & Run the top Linux command. This option may have {\ttfamily top} command option(s) separated by comma (,) \\
\hline
{\em -\/} & Simply ignored. This option can be used to avoid the ambiguity of the options.\\
\hline
\end{DoxyParams}
\begin{DoxyParagraph}{Description}
{\ttfamily pips} is a filter to target only Pi\-P tasks (including Pi\-P root) to show status like the way what the {\ttfamily ps} commands does and send signals to the selected Pi\-P tasks.
\end{DoxyParagraph}
Just like the {\ttfamily ps} command, {\ttfamily pips} can take the most familiar {\ttfamily ps} options {\ttfamily a}, {\ttfamily u}, {\ttfamily x}. Here is an example;

\begin{DoxyVerb}$ pips
PID   TID   TT       TIME     PIP COMMAND
18741 18741 pts/0    00:00:00 RT  pip-exec
18742 18742 pts/0    00:00:00 RG  pip-exec
18743 18743 pts/0    00:00:00 RL  pip-exec
18741 18744 pts/0    00:00:00 0T  a
18745 18745 pts/0    00:00:00 0G  b
18746 18746 pts/0    00:00:00 0L  c
18747 18747 pts/0    00:00:00 1L  c
18741 18748 pts/0    00:00:00 1T  a
18749 18749 pts/0    00:00:00 1G  b
18741 18750 pts/0    00:00:00 2T  a
18751 18751 pts/0    00:00:00 2G  b
18741 18752 pts/0    00:00:00 3T  a
\end{DoxyVerb}


here, there are 3 {\ttfamily pip-\/exec} root processes running. Four pip tasks running program 'a' with the ptherad mode, three Pi\-P tasks running program 'b' with the process\-:got mode, and two Pi\-P tasks running program 'c' with the process\-:preload mode.

Unlike the {\ttfamily ps} command, two columns 'T\-I\-D' and 'P\-I\-P' are added. The 'T\-I\-D' field is to identify Pi\-P tasks in pthread execution mode. three Pi\-P tasks running in the pthread mode. As for the 'Pi\-P' field, if the first letter is 'R' then that pip task is running as a Pi\-P root. If this letter is a number from '0' to '9' then this is a Pi\-P task (not root). The number is the least-\/significant digit of the Pi\-P I\-D of that Pi\-P task. The second letter represents the Pi\-P execution mode which is common with Pi\-P root and task. 'L' is 'process\-:preload,' 'C' is 'process\-:pipclone,', 'G' is 'process\-:got,' and 'T' is 'thread.'

The last 'C\-O\-M\-M\-A\-N\-D' column of the {\ttfamily pips} output may be different from what the {\ttfamily ps} command shows, although it looks the same. It represents the command, not the command line consisting of a command and its argument(s). More precisely speaking, it is the first 14 letters of the command. This comes from the Pi\-P's specificity. Pi\-P tasks are not created by using the normal {\ttfamily exec} systemcall and the Linux assumes the same command line with the pip root process which creates the pip tasks.

If users want to have the other {\ttfamily ps} command options other than 'aux', then refer to the {\ttfamily -\/-\/ps} option described below. But in this case, the command lines of Pi\-P tasks (excepting Pi\-P roots) are not correct.

\begin{DoxyItemize}
\item {\ttfamily -\/-\/root} ({\ttfamily -\/r}) Only the Pi\-P root tasks will be shown. \begin{DoxyVerb}$ pips --root
PID   TID   TT       TIME     PIP COMMAND
18741 18741 pts/0    00:00:00 RT  pip-exec
18742 18742 pts/0    00:00:00 RG  pip-exec
18743 18743 pts/0    00:00:00 RL  pip-exec
\end{DoxyVerb}
\end{DoxyItemize}
\begin{DoxyItemize}
\item {\ttfamily -\/-\/task} ({\ttfamily -\/t}) Only the Pi\-P tasks (excluding root) will be shown. If both of {\ttfamily -\/-\/root} and {\ttfamily -\/-\/task} are specified, then firstly Pi\-P roots are shown and then Pi\-P tasks will be shown. \begin{DoxyVerb}$ pips --tasks
PID   TID   TT       TIME     PIP COMMAND
18741 18744 pts/0    00:00:00 0T  a
18745 18745 pts/0    00:00:00 0G  b
18746 18746 pts/0    00:00:00 0L  c
18747 18747 pts/0    00:00:00 1L  c
18741 18748 pts/0    00:00:00 1T  a
18749 18749 pts/0    00:00:00 1G  b
18741 18750 pts/0    00:00:00 2T  a
18751 18751 pts/0    00:00:00 2G  b
18741 18752 pts/0    00:00:00 3T  a
\end{DoxyVerb}
\end{DoxyItemize}
\begin{DoxyItemize}
\item {\ttfamily -\/-\/family} ({\ttfamily -\/f}) All Pi\-P roots and tasks of the selected Pi\-P tasks by the {\ttfamily P\-A\-T\-T\-E\-R\-N} optional argument of {\ttfamily pips}. \begin{DoxyVerb}$ pips - a
PID   TID   TT       TIME     PIP COMMAND
18741 18744 pts/0    00:00:00 0T  a
18741 18748 pts/0    00:00:00 1T  a
18741 18750 pts/0    00:00:00 2T  a
$ pips --family a
PID   TID   TT       TIME     PIP COMMAND
18741 18741 pts/0    00:00:00 RT  pip-exec
18741 18744 pts/0    00:00:00 0T  a
18741 18748 pts/0    00:00:00 1T  a
18741 18750 pts/0    00:00:00 2T  a
\end{DoxyVerb}
 In this example, \char`\"{}pips -\/ a\char`\"{} (the -\/ is needed not to confused the command name {\ttfamily a} as the {\ttfamily pips} option) shows the Pi\-P tasks which is derived from the program {\ttfamily a}. The second run, \char`\"{}pips -\/-\/family a,\char`\"{} shows the Pi\-P tasks of {\ttfamily a} and their Pi\-P root ({\ttfamily pip-\/exec}, in this example).\end{DoxyItemize}
\begin{DoxyItemize}
\item {\ttfamily -\/-\/kill} ({\ttfamily -\/k}) Send S\-I\-G\-T\-E\-R\-M signal to the selected Pi\-P tasks. \item {\ttfamily -\/-\/signal} ({\ttfamily -\/s}) {\ttfamily S\-I\-G\-N\-A\-L} Send the specified signal to the selected Pi\-P tasks. \item {\ttfamily -\/-\/ps} ({\ttfamily -\/\-P}) This option may be followed by the {\ttfamily ps} command options. When this option is specified, the P\-I\-Ds of selected Pi\-P tasks are passed to the {\ttfamily ps} command with the specified {\ttfamily ps} command options, if given. \item {\ttfamily -\/-\/top} ({\ttfamily -\/\-T}) This option may be followed by the {\ttfamily top} command options. When this option is specified, the P\-I\-Ds of selected Pi\-P tasks are passed to the {\ttfamily top} command with the specified {\ttfamily top} command options, if given. \item {\ttfamily P\-A\-T\-T\-E\-R\-N} The last argument is the pattern(s) to select which Pi\-P tasks to be selected and shown. This pattern can be a command name (only the first 14 characters are effective), P\-I\-D, T\-I\-D, or a Unix (Linux) filename matching pattern (if the fnmatch Python module is available). \begin{DoxyVerb}$ pips - *-*
PID   TID   TT       TIME     PIP COMMAND
18741 18741 pts/0    00:00:00 RT  pip-exec
18742 18742 pts/0    00:00:00 RG  pip-exec
18743 18743 pts/0    00:00:00 RL  pip-exec
\end{DoxyVerb}
\end{DoxyItemize}
\begin{DoxyNote}{Note}
{\ttfamily pips} collects Pi\-P tasks' status by using the Linux's {\ttfamily ps} command. When the {\ttfamily -\/-\/ps} or {\ttfamily -\/-\/top} option is specified, the {\ttfamily ps} or {\ttfamily top} command is invoked after invoking the {\ttfamily ps} command for information gathering. This, however, may result some Pi\-P tasks may not appear in the invoked {\ttfamily ps} or {\ttfamily top} command when one or more Pi\-P tasks finished after the first {\ttfamily ps} command invocation. The same situation may also happen with the {\ttfamily -\/-\/kill} or -\/-\/signal option. 
\end{DoxyNote}
